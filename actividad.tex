\documentclass{article}
\usepackage{graphicx} % Required for inserting images

\title{Linear Search}
\author{Diego Avila}
\date{12 de Mayo, 2025}

\begin{document}

\maketitle

\section*{Que es linear search?}
\large Linear search es un algoritmo simple y efectivo, sirve para buscar cierto caractér o numero dentro de una lista, string, numero, etc. (en este caso lo hice con una lista), este algoritmo recibe lo que deseas encontrar dentro de la lista y en caso de que esté dentro de esta, retornara el indice donde se encuentra.

\section*{Quien lo creó?}
\large Este algoritmo no fue creado por una persona en particular, debido a que este es una idea muy basica pero aun asi es fundamental, se dice que este algoritmo se ha usado desde los inicios de la programación, de hecho, es una idea tan basica pero util, que se puede utilizar fuera de la programacion, por ejemplo buscando en registros.

\section*{De donde lo saqué?}
\large Lo encontre simplemente buscando en google, pese a que ya lo conocia estaba buscando una idea.

\end{document}
